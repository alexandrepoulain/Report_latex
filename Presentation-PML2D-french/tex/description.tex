\section{Description}
\subsection{Propagation des ondes élastiques dans des solides}
\begin{frame}{Equations}
\begin{itemize}
\item \underline{Système d'équations:}
\begin{equation}
\begin{cases}
\sum_{j}\frac{\partial \sigma_{ij}}{\partial x_j} = \rho \frac{\partial^2 u_i}{\partial t^2} \\
\sigma_{ij} = \sum_{k,l} C_{ijkl} \epsilon_{ij} \\
\epsilon_{ij} = \frac{1}{2}\left(\frac{\partial u_i}{\partial x_j} + \frac{\partial u_j}{\partial x_i} \right)
\end{cases}
\end{equation}
\item \underline{Coordonnées complexes: $x_i \rightarrow \tilde{x}_i: \mathbb{R} \rightarrow \mathbb{C}$}
\begin{equation}
\frac{\partial \tilde{x}_i}{\partial x_i} = \lambda_i(x_i) = 1+f_i^e(x_i)-i \frac{f^p_i(x_i)}{b k_s}
\end{equation}
$b$: Longueur caractéristique du problème.\\
$k_s = \frac{\omega}{c_s}$: nombre d'ondes.\\
$c_s$: Célérité des ondes de cisaillement.
 \end{itemize}
\end{frame}

\begin{frame}{Forme discrète des PML}
Après:
\begin{itemize}
\item Intégration sur le domaine de calcul.
\item Discrétisation en temps et en espace.
\end{itemize}
\begin{equation}
M \ddot{U}_{n+1} + (C+\tilde{C})\dot{U}_{n+1} + (K+\tilde{K})U_{n+1} + P(\epsilon_n, E_n, \Sigma_n) = F_{ext}
\end{equation}
with
\begin{columns}[T] % align columns
\begin{column}{.48\textwidth}
\begin{equation*}
m^{e} = \int_{\Omega_e} \rho f_m N_I N_J d\Omega_e I_d
\end{equation*}
\begin{equation*}
 c^{e} = \int_{\Omega_e} \rho f_c \frac{c_s}{b} N_I N_J d\Omega_e I_d
\end{equation*}
\begin{equation*}
  k^{e} = \int_{\Omega_e} \frac{\mu}{b^2} f_k N_I N_J d\Omega_e I_d 
\end{equation*}
\end{column}%
\hfill%
\begin{column}{.48\textwidth}
\begin{equation*}
\tilde{c}^{e} = \frac{1}{dt} \int_{\Omega_e} \tilde{B}^T D B^\epsilon d\Omega_e
\end{equation*}
\begin{equation*}
\tilde{k}^{e} = \frac{1}{dt} \int_{\Omega_e} \tilde{B}^T D B^Q d\Omega_e
\end{equation*}

\end{column}%
\end{columns}

\end{frame}


\begin{frame}{Forme discrète des PML}
\begin{equation*}
P^e(\epsilon_n, E_n, \Sigma_n) = \int_{\Omega_e} \tilde{B}^T \frac{D}{dt} \left[\frac{1}{dt}\hat{F}^{\epsilon} \hat{\epsilon} - \hat{F}^Q \hat{E}_n \right] + \tilde{B}^p \hat{\Sigma}_n d\Omega_e
\end{equation*}

\begin{equation*}
\begin{cases}
f_m = (1+f^e_1(x1))(1+f^e_2(x2))\\
f_c = (1+f^e_1(x1))f^p_2(x2) + (1+f^e_2(x2))f^p_1(x1)\\
f_k = f^p_1(x_1)f^p_2(x_2)
\end{cases}
\end{equation*}
Integrale de la contrainte et de la tension:
\begin{equation*}
\doubleunderline{\Sigma} = \int^t_0 \doubleunderline{\sigma} dt, \doubleunderline{E} = \int^t_0 \doubleunderline{\epsilon} dt
\end{equation*}
Fonctions d'atténuation:
\begin{equation*}
f^\alpha_x = a_\alpha\left(\frac{x - x_0}{L_p} \right)^n \hspace{1cm} f^\alpha_y = a_\alpha\left(\frac{y - y_0}{L_p} \right)^n
\end{equation*}
\end{frame}


