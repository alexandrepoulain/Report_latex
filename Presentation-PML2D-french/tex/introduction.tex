\section{Introduction}
\subsection{Bords absorbants}
\begin{frame}{Bords absorbants}
    \begin{itemize}[<+->]
        \item Méthode communément employée pour résoudre numériquement la propagation d'ondes en milieux infinis.
        \item Important sujet de recherche pour beaucoup de recherches et d'applications d'ingéniérie
        \item Ex: Simulation de séismes, structure des sols, géophysique, mesure dans le sous-sol, problème de guidage d'ondes.        \pause
        \vspace{0.7cm}
        \item 2 types de méthodes:
        \begin{itemize}[<+->]
        \pause
            \item \underline{Conditions de bords absorbants :} Conditions spécifiques aux bords du modèle.
        \pause
            \item \underline{Couches de bords absorbants :} Couche entourant le bord du domaine d'intérêt.
        \end{itemize}    
    \end{itemize}
\end{frame}

\subsection{Couches parfaitement adaptées (PML)}
\begin{frame}{Couches parfaitement adaptées}
\pause
    \begin{itemize}
        \item \underline{split-field PML:} 
        \begin{itemize}
        \pause
            \item Introduit par Béranger dans le contexte des ondes électromagnétiques
            \item Prolongé aux ondes élastodynamiques par Hastings.
            \item Transformation: prolongation analytique de l'équation d'onde dans le domaine complexe.
            \item Dans ce nouveau domaine, les ondes oscillantes (propagatives) sont remplacées par des ondes dont l'amplitude décroit de manière exponentielle.
            \item Split-field: séparation des variables en deux composantes parallèle et perpendiculaire dans la PML.
        \end{itemize}
    \item \underline{unsplit-field PML:} 
        \begin{itemize}
        \pause
            \item Introduite par Wang dans le contexte de l'élastodynamique (CPML) $\rightarrow$ \textcolor{red}{Complexe} 
            \pause
            \item Basu and Chopra \cite{Basu2003}: unsplit-field PML pour l'élastodynamique harmonique dans le temps $\rightarrow$ \textcolor{green}{implémentation éléments finis}
        \end{itemize}
    \end{itemize}
\end{frame}

% Frame d'explication des PML.


\begin{frame}{Stabilité des PML dans la littérature}
\pause
    \begin{itemize}
        \item \underline{split-field PML:} 
        \begin{itemize}
        \pause
        	\item Analyse de stablité en utilisant les diagrammes de lenteur (slowness diagrams) et les fronts d'ondes.
        	\item Définitions de conditions nécessaires et suffisantes pour la stabilité des PML \cite{Becache}. 
            \item discrétisation en différences finis de premier ordre: souffre d'instabilité pour les milieux anisotropes.
        \end{itemize}
    \item \underline{unsplit-field PML:} 
        \begin{itemize}
        \pause
            \item Stabilité prouvée pour l'électromagnetisme (équations de Maxwell).
            \pause
            \item Pour les problèmes de propagation d'ondes élastiques: peu d'informations.
        \end{itemize}
    \end{itemize}
\end{frame}

