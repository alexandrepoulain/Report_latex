\section{Stability of the integration method for the two-dimensional PML}
\subsection{Description}
First of all, the state vectors at time $t_n$ and $t_{n+1}$ are defined by
\begin{equation}
X_n = \begin{pmatrix}
\dot{u}_n \\
u_n \\
\hat{\epsilon}_{n-1} \\
\hat{E}_{n-1} \\
\hat{\Sigma}_{n-1}
\end{pmatrix}, \hspace{1cm}
X_{n+1}= \begin{pmatrix}
\dot{u}_{n+1} \\
u_{n+1} \\
\hat{\epsilon}_{n} \\
\hat{E}_{n} \\
\hat{\Sigma}_{n}
\end{pmatrix}
\end{equation}  
Since the considered element is a four-noded linear element, $\dot{u}_n$ and $u_n$ have a shape of $8 \times 1$. For each node we have a component along the x- and y-direction: for example for node $1$ we have $u_{n,1} = \begin{pmatrix}
u_{n,1,x}\\ u_{n,1,y}
\end{pmatrix} $. We expressed the strain, its integral and the integral of the stress in Voigt notation. For each point of quadrature, these elements will have $3$ components. Therefore in the state vector $\hat{\epsilon}_{n}$, $\hat{E}_{n}$ and $\hat{\Sigma}_{n}$ will have a size of $(ng*3) \times 1$ with $ng$ corresponding to the order of the quadrature.   
\par Let us recall the equation of motion in the PML at time $t_n$:
\begin{equation}
M \ddot{u}_n +\left(C+\tilde{C}\right)\dot{u}_n 
+\left(K+\tilde{K}\right)u_n + P(\epsilon_{n-1},E_{n-1},\Sigma_{n-1}) = F_{ext}
\label{eq:motion-pml-tn}
\end{equation}
which gives us the expresion for $M\ddot{u}_n$:
\begin{equation}
M \ddot{u}_n = F_{ext} -\left(C+\tilde{C}\right)\dot{u}_n 
-\left(K+\tilde{K}\right)u_n - P(\epsilon_{n-1},E_{n-1},\Sigma_{n-1})
\label{eq:motion-pml-tn-reorganized}
\end{equation}
And at time $t_{n+1}$:
\begin{equation}
M \ddot{u}_{n+1} = F_{ext} -\left(C+\tilde{C}\right)\dot{u}_{n+1} 
-\left(K+\tilde{K}\right)u_{n+1} - P(\epsilon_{n},E_{n},\Sigma_{n})
\label{eq:motion-pml-tn+1-reorganized}
\end{equation}
Let us recall the Newmark's relationships:
\begin{equation}
	\begin{cases}
		u_{n+1} = u_n + \Delta t \dot{u}_n + \Delta t^2 \left(\frac{1}{2}-\beta\right)\ddot{u}_n + \beta \Delta t^2 \ddot{u}_{n+1} \\
		\dot{u}_{n+1} = \dot{u}_n + \Delta t (1-\gamma) \ddot{u}_n + \gamma \Delta t \ddot{u}_{n+1}
	\end{cases}
	\label{eq:Newmark-relations}
\end{equation}
Now multiply the equations \ref{eq:Newmark-relations} by the mass matrix $M$ and use the equations \ref{eq:motion-pml-tn-reorganized} and \ref{eq:motion-pml-tn+1-reorganized} to obtain:
\begin{align}
		M u_{n+1} = M u_n + \Delta t M \dot{u}_n + &  \Delta t^2 \left(\frac{1}{2}-\beta\right)\left[ F_{ext} -\left(C+\tilde{C}\right)\dot{u}_n -\left(K+\tilde{K}\right)u_n - P(\epsilon_{n-1},E_{n-1},\Sigma_{n-1}) \right]  \notag\\ + &  \beta \Delta t^2 \left[ F_{ext} -\left(C+\tilde{C}\right)\dot{u}_{n+1} 
-\left(K+\tilde{K}\right)u_{n+1} - P(\epsilon_{n},E_{n},\Sigma_{n}) \right] 
	\label{eq:main-relations1}
\end{align}
\begin{align}
		M \dot{u}_{n+1} = M \dot{u}_n +  &\Delta t (1-\gamma) \left[ F_{ext} -\left(C+\tilde{C}\right)\dot{u}_n 
-\left(K+\tilde{K}\right)u_n - P(\epsilon_{n-1},E_{n-1},\Sigma_{n-1}) \right] \notag\\ + & \gamma \Delta t \left[ F_{ext} -\left(C+\tilde{C}\right)\dot{u}_{n+1} 
-\left(K+\tilde{K}\right)u_{n+1} - P(\epsilon_{n},E_{n},\Sigma_{n}) \right] 
	\label{eq:main-relations2}
\end{align}
In the above equations, the functional $P$ depends on the strain, its integral and the integral of the stress of the previous time step and take the following expression:
\begin{equation}
P(\hat{\epsilon_n},\hat{E_n},\hat{\Sigma_n}) = \int_{\Omega_e} \tilde{B}^T \frac{D}{\Delta t}\left[\frac{1}{\Delta t} \hat{F}^\epsilon \hat{\epsilon}_n - \hat{F}^Q \hat{E}_n  \right] + \tilde{B}^{p T} \hat{\Sigma}_n d\Omega_e
\label{eq:P-exp}
\end{equation}
Using a gaussian quadrature to evaluate numerically the integral:
\begin{equation}
P(\hat{\epsilon_n},\hat{E_n},\hat{\Sigma_n}) = \sum_{i=1}^{ng} \sum_{j=1}^{ng} \omega_i \omega_j \left[\tilde{B}^T(\eta_i,\xi_j) \frac{D}{\Delta t} \left[ \hat{F}(\eta_i,\xi_j)\hat{\epsilon}_n(\eta_i,\xi_j) - \hat{F}^Q(\eta_i,\xi_j) \hat{E}_n(\eta_i,\xi_j) \right]  + \tilde{B}^{pT}(\eta_i,\xi_j) \hat{\Sigma}_n(\eta_i,\xi_j)\right]
\label{eq:app-P}
\end{equation}  
The objective is to find a matrix formulation of \ref{eq:app-P}and in order to do so let us decompose the problem. First of all, we seek a matrix to be multiplied with $\hat{\epsilon}_n$ contained in the state vector $X_{n+1}$. We recall that the size of this component is $(3*ng)\times 1$. 
Let us define the following matrix $S$ with 
\begin{equation}
S_{i,j}=\omega_i \omega_j \left[ \tilde{B}^T(\eta_i,\xi_j) \frac{D}{\Delta t^2} \hat{F}^\epsilon(\eta_i,\xi_j)\right]
\label{eq:S}
\end{equation}
Since $\tilde{B}$ has a size of $3\times (2*ng^2)$, $D$: $3 \times 3$ and $\hat{F}^\epsilon$: $3\times (3*ng^2)$, $S$ has a shape of $(2*ng^2) \times (3*ng^2)$. Thus the matrix $S$ is defined by bloc in the following way for example if $ng=2$:
\begin{equation}
S = \left[ S_{1,1} S_{1,2} S_{2,1} S_{2,2} \right] 
\label{eq:S-constru}
\end{equation}
In this case the matrix will have a size of $8 \times 12$.
The same construction can be applied to the matrix $R$ with:
\begin{equation}
R_{i,j}= - \omega_i \omega_j \left[ \tilde{B}^T(\eta_i,\xi_j) \frac{D}{\Delta t} \hat{F}^Q(\eta_i,\xi_j)\right]
\label{eq:R}
\end{equation}
This matrix will be multiplied to the component $\hat{E}_n$ of the state vector $X_{n+1}$. We have the same sizes for the matrices $\tilde{B}$ and $\hat{F}^Q$ as previously. Thus $R$ will have a shape of $(2*ng^2) \times (3*ng^2)$. For $\hat{\Sigma}_n$, let us define the matrix $V$ with:
\begin{equation}
V_{i,j}= \omega_i \omega_j \tilde{B}^{pT}(\eta_i,\xi_j)
\label{eq:R}
\end{equation}
Since $\tilde{B}^p$ has size of $3\times (2*ng^2)$, the matrix $V$ with the same construction as the above matrices $R$ and $S$ will have a shape of $(2*ng^2)\times (3*ng^2)$. 
\par To the equations \ref{eq:main-relations1} and \ref{eq:main-relations2}, we also add the following relations. From the expression of the strain at the next time step:
\begin{equation}
\hat{\epsilon}_{n+1} = \frac{1}{\Delta t} \left[B^\epsilon \dot{u}_{n+1} + B^Q u_{n+1} + \frac{1}{\Delta t} \hat{F}^\epsilon \hat{\epsilon}_n-\hat{F}^Q \hat{E}_n \right]
\label{eq:stain-tn+1}
\end{equation}
We can derived the following expression:
\begin{equation}
- \frac{1}{\Delta t} B^\epsilon \dot{u}_{n+1} - \frac{1}{\Delta t} B^Q u_{n+1} + \hat{\epsilon}_{n+1}-\frac{1}{\Delta t^2} \hat{F}^\epsilon \hat{\epsilon}_n + \frac{1}{\Delta t} \hat{F}^Q \hat{E}_n = 0
\label{eq:main-relations3}
\end{equation} 
From the recurence relationship of the integral of the strain we can derived the following expression:
\begin{equation}
\hat{E}_{n+1} - \hat{E}_n - \Delta t \hat{\epsilon}_{n+1} = 0
\label{eq:main-relations4}
\end{equation} 
And the relationship for the integral of the stress:
\begin{equation}
\hat{\Sigma}_{n+1} - \hat{\Sigma}_n - \Delta t D \hat{\epsilon}_{n+1} = 0
\label{eq:main-relations5}
\end{equation} 
\par Using the equations \ref{eq:main-relations1}, \ref{eq:main-relations2}, \ref{eq:main-relations3},\ref{eq:main-relations4} and \ref{eq:main-relations5}, we can formulate the 2 matrices $A$ and $B$ such that:
\begin{equation}
A X_{n+1} + B X_n = 0
\label{eq:geradin-ampl}
\end{equation} 
Therefore we have:
\begin{equation}
A = \begin{bmatrix}
\Delta t^2 \beta M^{-1} (C+\tilde{C}) & Id + \Delta t^2 \beta M^{-1}(K+\tilde{K}) & \Delta t^2 \beta M^{-1} S & \Delta t^2 \beta M^{-1} R & \Delta t^2 \beta M^{-1} V \\
I_d + \Delta t \gamma M^{-1} (C+\tilde{C})& \Delta t \gamma M^{-1} (K+\tilde{K}) & \Delta t \gamma M^{-1} S& \Delta t \gamma M^{-1} R & \Delta t \gamma M^{-1} V \\
-\frac{1}{\Delta t} B^\epsilon & -\frac{1}{\Delta t} B^Q & I_d & 0 & 0 \\
0 & 0 & -\Delta t I_d & I_d & 0 \\
0 & 0 & -\Delta t L & 0 & I_d 
\end{bmatrix}
\label{eq:A}
\end{equation}
In this matrix $A$ on the last "row", the matrix $L$ is a squared matrix such that   
on the diagonal we have blocs of size  $3 \times 3$ correspnding to the matrix $D$. Here the identity matrix $I_d$ is of size $3*(ng^2) \times 3*(ng^2)$.
We also have the definition of the matrix $B$.
\begin{equation}
\resizebox{\columnwidth}{!}{
$
B = \begin{bmatrix}
 - \Delta t I_d + \Delta t^2 (\frac{1}{2} - \beta) M^{-1} (C+\tilde{C}) & - Id + \Delta t^2 (\frac{1}{2} - \beta) M^{-1}(K+\tilde{K}) & \Delta t^2 (\frac{1}{2} - \beta) M^{-1} S & \Delta t^2 (\frac{1}{2} - \beta) M^{-1} R & \Delta t^2 (\frac{1}{2} - \beta) M^{-1} V \\
-I_d + \Delta t (1-\gamma) M^{-1} (C+\tilde{C})& \Delta t (1-\gamma) M^{-1} (K+\tilde{K}) & \Delta t (1-\gamma) M^{-1} S& \Delta t (1-\gamma) M^{-1} R & \Delta t (1-\gamma) M^{-1} V \\
0 & 0 & -\frac{1}{\Delta t^2}\hat{F}^\epsilon & \frac{1}{\Delta t} \hat{F}^Q & 0 \\
0 & 0 & 0 & -I_d & 0 \\
0 & 0 & 0 & 0 & -I_d 
\end{bmatrix}
$
}
\label{eq:B}
\end{equation}


\subsection{Numerical results}