\section{Stability of the integration method for the two-dimensional PML}
First of all, the state vectors at time $t_n$ and $t_{n+1}$ are defined by
\begin{equation}
X_n = \begin{pmatrix}
\dot{u}_n \\
u_n \\
\hat{\epsilon}_{n-1} \\
\hat{E}_{n-1} \\
\hat{\Sigma}_{n-1}
\end{pmatrix}, \hspace{1cm}
X_{n+1}= \begin{pmatrix}
\dot{u}_{n+1} \\
u_{n+1} \\
\hat{\epsilon}_{n} \\
\hat{E}_{n} \\
\hat{\Sigma}_{n}
\end{pmatrix}
\end{equation}  
Since the element considered is a four-noded linear element $\dot{u}_n$ and $u_n$ have a shape of $8 \times 1$. For each node we have a component along the x- and y-direction: for example for node $1$ we have $u_{n,1} = \begin{pmatrix}
u_{n,1,x}\\ u_{n,1,y}
\end{pmatrix} $. We expressed the strain, its integral and the integral of the stress in Voigt notation. For each point of quadrature, these elements will have $3$ components. Therefore in the state vector $\hat{\epsilon}_{n}$, $\hat{E}_{n}$ and $\hat{\Sigma}_{n}$ will have a size of $(ng*3) \times 1$ with $ng$ corresponding to the order of the quadrature.   
\par Let us recall the equation of motion in the PML at time $t_n$:
\begin{equation}
M \ddot{u}_n +\left(C+\tilde{C}\right)\dot{u}_n 
+\left(K+\tilde{K}\right)u_n + P(\epsilon_{n-1},E_{n-1},\Sigma_{n-1}) = F_{ext}
\label{eq:motion-pml-tn}
\end{equation}
which gives us the expresion for $M\ddot{u}_n$:
\begin{equation}
M \ddot{u}_n = F_{ext} -\left(C+\tilde{C}\right)\dot{u}_n 
-\left(K+\tilde{K}\right)u_n - P(\epsilon_{n-1},E_{n-1},\Sigma_{n-1})
\label{eq:motion-pml-tn-reorganized}
\end{equation}
And at time $t_{n+1}$:
\begin{equation}
M \ddot{u}_{n+1} = F_{ext} -\left(C+\tilde{C}\right)\dot{u}_{n+1} 
-\left(K+\tilde{K}\right)u_{n+1} - P(\epsilon_{n},E_{n},\Sigma_{n})
\label{eq:motion-pml-tn+1-reorganized}
\end{equation}
Let us recall the Newmark relationships:
\begin{equation}
	\begin{cases}
		u_{n+1} = u_n + \Delta t \dot{u}_n + \Delta t^2 \left(\frac{1}{2}-\beta\right)\ddot{u}_n + \beta \Delta t^2 \ddot{u}_{n+1} \\
		\dot{u}_{n+1} = \dot{u}_n + \Delta t (1-\gamma) \ddot{u}_n + \gamma \Delta t \ddot{u}_{n+1}
	\end{cases}
	\label{eq:Newmark-relations}
\end{equation}
Now multiply the equations \ref{eq:Newmark-relations} by the mass matrix $M$ and use the equations \ref{eq:motion-pml-tn-reorganized} and \ref{eq:motion-pml-tn+1-reorganized} to obtain:
\begin{align}
		M u_{n+1} = M u_n + \Delta t M \dot{u}_n + &  \Delta t^2 \left(\frac{1}{2}-\beta\right)\left[ F_{ext} -\left(C+\tilde{C}\right)\dot{u}_n -\left(K+\tilde{K}\right)u_n - P(\epsilon_{n-1},E_{n-1},\Sigma_{n-1}) \right]  \notag\\ + &  \beta \Delta t^2 \left[ F_{ext} -\left(C+\tilde{C}\right)\dot{u}_{n+1} 
-\left(K+\tilde{K}\right)u_{n+1} - P(\epsilon_{n},E_{n},\Sigma_{n}) \right] 
	\label{eq:main-relations1}
\end{align}
\begin{align}
		M \dot{u}_{n+1} = M \dot{u}_n +  &\Delta t (1-\gamma) \left[ F_{ext} -\left(C+\tilde{C}\right)\dot{u}_n 
-\left(K+\tilde{K}\right)u_n - P(\epsilon_{n-1},E_{n-1},\Sigma_{n-1}) \right] \notag\\ + & \gamma \Delta t \left[  \dot{u}_n + \Delta t (1-\gamma) \ddot{u}_n + \gamma \Delta t \ddot{u}_{n+1} \right]
	\label{eq:main-relations2}
\end{align}
In the above equations, the functional $P$ depends on the strain, its integral and the integral of the strain of the previous time step and take the following expression:
\begin{equation}
P(\hat{\epsilon_n},\hat{E_n},\hat{\Sigma_n}) + \int_{\Omega_e} \tilde{B}^T \frac{D}{\Delta t}\left[\frac{1}{\Delta t} \hat{F}^\epsilon \hat{\epsilon}_n - \hat{F}^Q \hat{E}_n  \right] \tilde{B}^{p T} \hat{\Sigma}_n d\Omega_e
\label{eq:P-exp}
\end{equation}

