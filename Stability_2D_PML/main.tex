\documentclass[11pt]{article}
\usepackage[utf8]{inputenc}
\usepackage[T1]{fontenc}
\usepackage[french]{babel}
\usepackage{amsmath, amsfonts, amssymb}
\usepackage{verbatim}
\usepackage{amsthm}
\usepackage{geometry}
\usepackage{array}
\usepackage{caption}
\usepackage{graphicx}
\usepackage{float} 
\usepackage{hyperref}
\usepackage{adjustbox}
\usepackage{nomencl}
\usepackage[]{algorithm2e}


% taille des marges
\geometry{hmargin=2cm,vmargin=2.5cm}

\title{Perfectly Matched Layer 1D}
\author{Poulain Alexandre}
\date{\today}

\begin{document}
\pagestyle{empty}
\begin{titlepage}
\begin{sffamily}
\begin{center}
    \vspace*{2cm}
    \noindent\hrulefill \\
    \vspace*{1cm}
    {\huge \bfseries Absorbing Boundary Layers in Time Domain Elastodynamics: }\\[0.5cm]
    {\huge  \itshape Two-Dimensional Perfectly Matched Layer} \\[0.5cm]
    {\large Computational Solid Mechanics Laboratory  (LSMS)}\\[0.5cm]
    {\large Ecole Polytechnique Fédérale de Lausanne 2017-2018}\\[1cm]
    \noindent\hrulefill \\
    \vspace{2cm}
    March 2018 - August 2018 \\[1cm]
    


      \includegraphics{images/EPFL-Logo.jpg}\\[1cm]
      

    \emph{Student:} Alexandre \textsc{Poulain}\\[1cm]
    \emph{Project supervisor:} Michael \textsc{Brun}\\[1cm]
    \emph{Laboratory director:} Jean-François \textsc{Molinari}

    
    
    \vfill{\large April 2018}

\end{center}
\end{sffamily}
\end{titlepage}

\newpage
\renewcommand{\abstractname}{Abstract}
\begin{abstract}
%% Background (short)
% What is known
Absorbing boundary layers are widely employed to solve numerically wave propagation phenomena in infinite or unbounded domains. Several techniques exist to construct these boundaries such as Rayleigh damping in the absorbing layer or as presented in this report, perfectly matched layers (PML).  
% What is not known
Since the construction of these PMLs includes many aspects, this remains a challenge and several past researches lead for example in the case of the split-field PML for anisotropic problem to unstable solutions.  We attend in this report to present a stable one dimensional PML based on the weak form of the equations of elastodynamics. 
%% Methods (long)
% What was done
First of all, the presentation of the construction of these PML will be detailed, including an extensive description of the equations governing this latter. The second part will be dedicated to the numerical results for test cases: a special attention will be dedicated to the reflection of the waves. We will shortly see that the PML is efficient to attenuate incident wave with less than $1$ percent of the incident wave reflected in some extreme cases where the length of the PML and the coefficients of the attenuation function are reduced to their minimum. The last part will describe the proof of the stability: the stability of the numerical scheme will be proved first and the theoretical stability will be reviewed after that since stability of one doesn't lead to stability of the other.    
%% Results


%% Conclusions




\end{abstract}


\newpage

\tableofcontents

\newpage
\makenomenclature
 
\mbox{}
 
\nomenclature{$c_s$}{S-wave velocity}
\nomenclature{$E$}{Young modulus}
\nomenclature{$\sigma$}{Stress}
\nomenclature{$\Sigma$}{Integral of stress}
\nomenclature{$\epsilon$}{Strain}
\nomenclature{$E$}{Integral of strain}
\nomenclature{$u$,$\Dot{u}$,$\Ddot{u}$}{Displacement, velocity and acceleration}
\nomenclature{$\lambda(x)$}{Stretching function}
\nomenclature{$f^p$, $f^e$}{Damping functions}
\nomenclature{$L$,$L_p$}{Length of medium, length of PML}
\nomenclature{$L_{ex}$,$L_{ey}$}{Element length in x- and y-direction}
\nomenclature{$\omega$}{Pulsation}
\nomenclature{$\eta$,$\ksi$}{Gauss points for numerical quadrature}
\nomenclature{$dt$}{Time step}
\nomenclature{$A$,$B$}{Amplification matrices}
\nomenclature{$\gamma$,$\beta$}{Newmark parameters}
\nomenclature{$M$,$C$,$\tilde{C}$,$K$,$\tilde{K}$}{Mass, damping, effective damping, stiffness, effective stiffness matrices (respectively)}
 
\printnomenclature

\newpage


\pagestyle{plain}
\setcounter{page}{1}
\section*{Introduction}
%% Absorbing boundary layers
% Description and historic 


The study of wave propagation in unbounded media is an growing topic of research and takes its applications in many engineering fields. 
The solution of the elastodynamic problem using infinite domains is interesting for the simulation of earthquake ground motion or for soil-structure problems where the reflection of the wave at the boundaries needs to be eliminated. 
To address this problem, a common practice is to add a layer surrounding the domain of interest in order to absorb the outgoing wave.   \\
% PMLs detailed historic and overview of the methods
Different methods have been developed to simulate wave propagation in unbounded media: the infinite elements methods, originally developed by Ungless \cite{Ungless} and Bettess \cite{Bettess}, which is close to the concept of finite elements but adds a new formulation including an infinite extent of the element region and shape functions. This method allows the approximation of the decaying laws governing the waves radiation process at infinity. The technique used here is to use finite element with their end nodes placed at infinity. The issues encountered with such technique is the same as the following method.  The appropriate absorbing boundary conditions is a method involving specific conditions at the model boundaries to approximate the radiation condition for the elastic waves \cite{Engquist}. The definition of these boundaries using this method leads to non stable scheme and spurious reflections cannot be avoided. These conditions are also not useful for practical calculations since they involve complex system of equations. The two first methods presented here presents the same drawbacks in terms of computation and analysis. However in order to solve these problems other methods have been developed leading to more efficient schemes.\\ 
As presented in this report, another method is to define a new layer to the simulation: An absorbing layer. The Rayleigh Damping layers is based on a Rayleigh/Caughey damping formulation to express the damping matrix $[C]$ in the classic formulation of elastodynamic problems. This damping matrix in the Rayleigh formulation can be expressed as a combination of stiffness and mass matrix. This damping matrix for finite elements is often already available in existing Finite Element software which makes this method really practical to use. In \cite{Semblat}, the efficiency of the method is given and shows a satisfactory behaviour of the method in terms of efficiency for the one and the two dimensional case. \\
Perfectly matched layers is another absorbing boundary layers method, that absorbs almost perfectly incident waves without any reflection from the truncation interface for all angles of incidence and frequencies. The wave entering into the PML decays with distance according to a user-defined decay function. The property of the non-reflection at the truncation interface is true in theory for the continuum case. Once a spatial discretization is used, numerical reflections are present but they can be attenuated using the parameters of the PML. These user-defined parameters can also increase the accuracy of the scheme used and even reduce the computational cost. Perfectly matched layers is a concept first introduced by Bérenger for the simulation of electromagnetic waves \cite{Berenger}. He used a split-field formulation and it arises from the use of complex-valued coordinate stretching in the electromagnetic wave equations \cite{Chew}. The field-splitting formulation permits to avoid convolutional operations in the time domain when the resulting forms are inverted back into the frequency domain and it is based on the partition of the variables into two components: parallel and perpendicular to the truncation boundary.  The drawback of this technique is that it alters the structure of the underlying differential equations and thus increases the number of unknowns. Another problem with Bérenger's split-field PML is that the problem is only weakly well-posed and thus prones to instability \cite{Abarbanel2}. This led to the development of strongly well-posed unsplit formulation \cite{Abarbanel1} but it turns out that these formulations also suffer from instability and need further manipulation of the equations to ensure it \cite{Abarbanel3}. However, the PML have been adapted for other linear wave equations such as the Helmholtz equation (scalar wave equation) \cite{Qi,Turkel,Harari}, linearised Euler equations \cite{Hu} or for the wave propagation in poroelastic materials \cite{Zeng}. An extensive discussion of these different methods is beyond the scope of this report since we focus on elastodynamic problems. \\
% Spit-field elasto
Indeed, the concept of PML was first adapted to elastodynamic wave propagation problems by Hastings et al \cite{Hastings}. This formulation was obtained by taking the split-field formulation of Bérenger and directly applying it to the P- and S-wave potentials. This formulation was obtained in term of displacement potentials and yields to a velocity-stress finite-difference method. The proof of the absorptive property of the PML was developed by Chew and Liu \cite{ChewLiu}: they developed in the same time a new split-field formulation for isotropic media using complex-valued coordinate stretching to obtain the equations governing the PML. Following the same idea Liu \cite{Liu} introduced a split-field PMLs for time-dependent elastic waves in cylindrical and spherical coordinates. Other split-field, time-domain PMLs for the velocity–stress formulation have been obtained and we refer to \cite{Zhang,Collino,Becache2} for the details of these methods and the presentation of a finite-differences-time-domain (FDTD) implementation of them. Another split-field formulation was introduced by Komatitsch and Tromp \cite{Komatitsch} where the stress term is eliminated and the displacement is split into four components. This results in a third-order in time semi-discrete forms for the four displacement fields or can be expressed by a second-order system coupled with one first-order equation for one of the displacement field. The main drawback of their method is its complexity but it is the first displacement only formulation for elastodynamics.\\
% unsplit PML
As we have seen before split-field PML suffer from instability since they are weakly well-posed. Wang \cite{Wang} introduced an unsplit formulation for finite-difference modeling of elastic wave propagation using convolution features (CPML). In contrast of the original formulation of CPML from the electromagnetic where they used complex-frequency-shifted stretching functions \cite{Teixeira, Roden}, Wang used standard stretching functions for its PML implementation.  In this report, we will develop an unsplit formulation using the finite element framework in the same spirit as Basu and Chopra. In \cite{Basu2003}, they introduced an unsplit-field PML for time-harmonic elastodynamics in 2D media. In \cite{Basu2004}, they developed the time-domain implementation of their PML and in \cite{Basu2008}, Basu extended its 2D formulation to 3D media using an explicit scheme. \\ 
Based on a decomposition of the elastodynamics equations as a first-order system, Cohen and Fauqueux \cite{Cohen} derived a split-field formulation where the strain tensor is split and they had to introduce independent stress variables to account for the split strain tensor components. This method was implemented using a mixed finite element approach and spectral elements. A different formulation was obtained by Festa and Vilotte \cite{Festa} where they followed classic lines for reducing the second-order displacement-only elastodynamic problem to a first-order in time system. Instead of slitting the strain tensor, they used split-fields for both the velocity and stress components. In the framework of unsplit PML, Drossaert and Giannopoulos \cite{Drossaert1} described an alternative implementation based on recursive integration (RIPML). But this implementation presents less performance than the CPML for elastodynamics using the complex-frequency-shifted stretching functions\cite{Drossaert2}. Meza-Fajardo and Papageorgiou \cite{Meza} discussed a novel PML approach. In the standard approach of PML the coordinate-stretching and associated decay functions are used along the direction normal to the PML interface. Meza-Fajardo and Papageorgiou introduced them along all coordinate directions resulting in a split-field, non-convolutional M-PML which shows superior performance compared to standard PML.  \\
All this literature survey of the different formulation of the PML can be summarised in the following table borrowed from \cite{Kucukcoban}:
\begin{table}[H]
    \centering
    \begin{tabular}{c|c|c}
        Implementation & split-field & unsplit-field \\
        \hline
        FD & Chew and Liu \cite{ChewLiu} & Wang and Tang \cite{Wang}\\
         & Hastings et al. \cite{Hastings} & Drossaert and Giannopoulos  \cite{Drossaert1, Drossaert2}\\
         &Liu\cite{Liu} & Komatitsch and Martin \cite{Komatitsch}\\
         &Collino and Tsogka \cite{Collino} & \\
         \hline
        FE/SE &  Bécache et al. \cite{Becache2} & Basu and Chopra \cite{Basu2003}\\
        & Komatitsch and Tromp \cite{Komatitsch}& Basu \cite{Basu2004}\\
        & Cohen and Fauqueux  \cite{Cohen} & \\
        & Festa and Vilotte \cite{Festa} & \\
        & Meza-Fajardo and Papageorgiou \cite{Meza} & 
    \end{tabular}
    \caption{PML implementations in time-domain elastodynamics}
    \label{tab:litterature}
\end{table}
The stability of the PML have been studied for mostly isotropic cases but Collino and Tsogka \cite{Collino} showed that the split-field standard PML is adequate in the case of anisotropic conditions. Also Bécache \cite{Becache} studied the stability of the PML and the effect of anisotropy: She showed that the standard PML is stable for isotropic cases and conditionally unstable for anisotropic applications. Bécache also proposed necessary conditions for stability in the form of inequalities choice of the stretching function.\\
In the light of these previous works we will attend in this report to describe the formulations of a two dimensional unsplit-field displacement-based PML. The implementations will be realised in the framework of finite elements. First of all we will describe in details the construction of the equations of the PML and also its formulation in the context of finite elements. Furthermore, we will discuss the stability of the PML: the numerical stability will be presented and the results will be analyzed since the the method used to obtained the proof of the stability permits to investigate other property of the scheme. In the last part we will review the numerical results obtained on two test cases. The first one one is a simple bar and the second is the Lamb test.






% Pros and cons

%% our work

% Objectives of our work


% Scientific importance of our work


% Originality of our work



\newpage
\section{Description}
%%%%%%
% Maybe a part here on how to make the junction GC maybe 

\subsection{Elastic medium}
As an introduction of the governing equations, let us decribe the elastic medium since the PML is surrounding it. In fact the construction of the PML relies on the same governing equations as the physical medium and share some parameters with it.\\
Let us consider an isotropic homogeneous elastic medium under plane-strain motion and in without body forces. The motion in the medium is governed by:
\begin{equation}
\begin{cases}
\sum_{j} \frac{\partial \sigma_{ij}}{\partial x_j} & =  \rho \ddot{u}_i \\
\sigma_{ij} &= \sum_{k,l} C_{ijkl} \epsilon_{kl} \\
\epsilon_{ij} &= \frac{1}{2} \left(\frac{\partial u_{i}}{\partial x_j} + \frac{\partial u_{j}}{\partial x_i} \right)
\end{cases}
\label{eq:2D-motion-elastic}
\end{equation} 
With $u(x,t)$ the displacement, $\epsilon$ the strain, $\sigma$ the stress, $\rho$ the mass density of the medium. $C$ is the material stiffness tensor and its components are expressed in term of the Kronecker delta $\delta_{ij}$ as:
\begin{equation}
C_{ijkl} = (\kappa-\frac{2}{3}\mu)\delta_{ij}\delta_{kl}+\mu(\delta_{ik}\delta_{jl} + \delta_{il} \delta_{jk})
\end{equation} 
Where $\kappa$ is the bulk modulus and $\mu$ the shear modulus.\\
If we consider an unbounded domain, the system \ref{eq:2D-motion-elastic} admits solutions of the form of P- waves and S-waves. The solutions of P-waves form have the following formulation: 
\begin{equation}
u(x,t) = q exp(-i k_p x . p) exp(i\omega t)
\label{eq:elastic-sol-p}
\end{equation}
with $k_p = \frac{\omega}{c_p}$ and $c_p = \sqrt{(\kappa+4\mu/3)/\rho}$ which is the celerity of the P-wave. In the equation \ref{eq:elastic-sol-p} $p$ denotes the unit vector pointing the direction of propagation of the wave and $q=\pm p$.   
The solution of S-wave form take the following formulation:
\begin{equation}
u(x,t) = q exp(-i k_s x.p) exp(i \omega t)
\end{equation}
Where $k_s = \omega / c_s$ and the S-wave speed is $c_s =\sqrt{ \mu/\rho}$ and $q.p =0$.
\subsection{Strong form in frequency domain}
Before describing the governing equations of the two-dimensional PML let us introduce the following notations: $\Omega_{PML}$ will denote the PML domain which is bounded by $\Gamma{PML} = \Gamma_{PML}^D \cup \Gamma_{PML}^N$. $\Gamma_{PML}^D$ corresponds to the boundary where the Dirichlet condition will be applied (imposed displacement). $\Gamma_{PML}^N$ is the boundary where the tractions will be applied and represents the boundary of the Neumann conditions. The intersection of these two boundary defined by their imposed conditions is null: $\Gamma_{PML}^D \cap \Gamma_{PML}^N = \emptyset$. The temporal domain will be denoted by $J = [0,T]$ with $T$ the end time.\\
The classical formulation of PML begins with the introduction of the complex-valued coordinates stretching functions $\lambda_i$. They are used to replace the real coordinates by the complex ones: $x_i \rightarrow \tilde{x}_i: \mathbb{R} \rightarrow \mathbb{C}$.
\begin{equation}
\lambda_i(x_i) = \frac{\partial \tilde{x}_i}{\partial x_i} = 1+f^e_i(x_i)-\frac{i}{b k_s} f^p_i(x_i)
\label{eq:complex-stret-2D}
\end{equation} 
where $b$ is the characteristic length of the problem. $k_s=\omega / c_s$ denotes the wavenumber ($\omega$ is the pulsation and $c_s$ is the celerity of shear waves). $f^p_i$ and $f^e_i$ are the attenuation functions for respectively propagating and evanescent waves. They are written as polynomial of order $n$. 
\begin{equation}
f^\alpha_i = a_\alpha \left(\frac{x_i - x_0}{L_p}  \right), x_i \in [x0,x0+d] \\
\label{eq:attenuation-functions}
\end{equation}
% Little explanation of what is the effect of the coordinate stretching function ? 
The tunable property of the attenuation function relies mainly on the formulation of the attenuation functions. The value of the coefficients of attenuation $a_p$, $a_e$, the order of the polynomial $n$, and the size of the PML $L_p$ can be defined by the user. As we will see in the section concerning the numerical results these parameters need to be chosen carefully depending on the problem to obtain the "best" result possible. The concept of "best result" depends of course of the expectations of the user. Accuracy versus performance is the recurent dilemma for all numerical simulations. The perfectly matched layer does not get out of this rule. Moreover an increase of the attenuation coefficients leads to a higher reflection of the wave from the interface between the physical medium and the PML. In order to choose the value these coefficients, we can use the reflection coefficient for an incident pressure wave given by \cite{Basu2004}:
\begin{equation}
R_{pp} = \frac{cos(\theta+\theta_s)}{cos(\theta-\theta_s)} exp\left[-2\frac{c_s}{c_p}F_1(L_p)cos(\theta)\right]
\label{eq:Rpp} 
\end{equation} 
Where $c_p$ stands for the velocity of P-waves. The incident P-wave is characterised by $\theta$ its angle of incidence and $\theta_s$ its reflective angle after being reflected at the end of the PML. $F_1$ corresponds to the integral over the PML of the attenuation function for propagating waves: $F_1(L_p) = \int_{s=0}^{L_p} f^p(s) ds = \frac{\beta_0 L_p}{n+1}$. Thus, the attenuation coefficients can be expressed in function of the reflection coefficient:
\begin{equation}
a_\alpha = \ln\left(\frac{cos(\theta+\theta_s)}{R_{pp}cos(\theta-\theta_s)} \right) \frac{c_p}{c_s} \frac{n+1}{L_p cos(\theta)}
\end{equation} 
If we consider that the incident wave has an angle of $\theta = \theta_s = 0$ therefore the value of the coeffecient has the form:
\begin{equation}
a_\alpha = \ln\left(\frac{1}{R_{pp}} \right) \frac{c_p}{c_s} \frac{n+1}{L_p cos(\theta)}
\end{equation}
\par Using the complex-valued coordinates stretching functions \ref{eq:complex-stret-2D}, the strong form of the equation of motion for the PML in the frequency domain is defined by:
\begin{equation}
\begin{cases}
\sum_{j} \frac{1}{\lambda_j(x_j)} \frac{\partial \sigma_{ij}}{\partial x_j} & = - \omega^2 \rho u_j \\
\sigma_{ij} &= \sum_{k,l} C_{ijkl} \epsilon_{ij} \\
\epsilon_{ij} &= \frac{1}{2} \left(\frac{1}{\lambda_j(x_j)} \frac{\partial u_{i}}{\partial x_j} + \frac{1}{\lambda_i(x_i)} \frac{\partial u_{j}}{\partial x_i} \right)
\end{cases}
\label{eq:2D-PML-freq}
\end{equation} 
Where $C_{ijkl}$ are the components of the elastic constitutive tensor. In fact the system \ref{eq:2D-PML-freq} defines a perfectly matched medium (PMM) and the elastic medium is just a specific case of PMM with $\lambda_j(x_j) = 1$, $\forall x \in \Omega_{PML}$. If we imagine a PML surrounding an elastic medium, as in the figure \ref{}, we have to define for the PML $\lambda_j(x_j) = 1$ for $\forall x \in \Gamma_{PML}$ at the interface between the two-subdomains. 
 
\subsection{Strong form in time domain}
\par The strong form of the PML in the temporal domain can be obtained by inverse Fourier transform. Indeed the introduction of the complex-valued coordinates stretching functions makes the application of this inverse easier. In the following equation the number of lines below a tensor will precise its order. 
%% Explanation on how to obtain these equations (for final version)
%First of all, the third equation in \ref{eq:2D-PML-freq} is premultiplied by $i \omega \Lambda^{-1}$ where
%\begin{equation}
%\Lambda = 
%\end{equation}
\begin{equation}
\begin{cases}
div(\doubleunderline{\sigma}\tilde{F}^e + \doubleunderline{\Sigma}\tilde{F}^p) = \rho f_m \underline{\ddot{u}} + \rho \frac{c_s}{b} f_c \underline{\dot{u}} + \frac{\mu}{b^2}f_k \underline{u}, & \text{In } \Omega_{PML} \times J\\
\doubleunderline{\sigma} =  C : \doubleunderline{\epsilon} , & \text{In } \Omega_{PML} \times J\\
F^{eT} \doubleunderline{\dot{\epsilon}}F^e + F^{pT}\underline{\epsilon}F^e + F^{eT}\doubleunderline{\epsilon}F^p + F^{pT} \doubleunderline{E} F^p = ...\\
\frac{1}{2}(\nabla \underline{\dot{u}}^T F^e + F^{eT} \nabla \underline{\dot{u}})+\frac{1}{2}(\nabla \underline{u}^T F^p + F^{pT} \nabla \underline{u}), & \text{In } \Omega_{PML} \times J\\
\end{cases}
\label{eq:2D-PML-strong-timeD}
\end{equation}
submitted to the homogeneous boundary conditions:
\begin{equation}
\begin{cases}
\underline{u}=0,& \text{on } \Gamma_{PML}^D\\
(\doubleunderline{\sigma}\tilde{F}^e + \doubleunderline{\Sigma}\tilde{F}^p).n , & \text{on } \Gamma_{PML}^N 
\end{cases}
\label{eq:conditions-lim}
\end{equation}
Let us now summarize the form of the different matrices $F^e,F^p,\tilde{F}^e$ and $\tilde{F}^p$ in the above equations.
\begin{equation}
F^e = \begin{bmatrix}
1+f^e_1(x1)&0\\0&1+f^e_2(x2)
\end{bmatrix}, F^p = \begin{bmatrix}
\frac{c_s}{b}f^p_1(x1)&0\\0&\frac{c_s}{b}f^p_2(x2) 
\end{bmatrix} 
\end{equation}
\begin{equation}
\tilde{F}^e = \begin{bmatrix}
1+f^e_2(x2)&0\\0&1+f^e_1(x1)
\end{bmatrix}, \tilde{F}^p = \begin{bmatrix}
\frac{c_s}{b}f^p_2(x2)&0\\0&\frac{c_s}{b}f^p_1(x1) 
\end{bmatrix} 
\end{equation}
Let is now focus on the first equation of \ref{eq:2D-PML-strong-timeD}, the functions $f_m$, $f_c$ and $f_k$ depend on the attenuation function and take the form:
\begin{equation}
\begin{cases}
f_m = (1+f^e_1(x1))(1+f^e_2(x2))\\
f_c = (1+f^e_1(x1))f^p_2(x2) + (1+f^e_2(x2))f^p_1(x1)\\
f_k = f^p_1(x_1)f^p_2(x_2)
\end{cases}
\end{equation}
The next elements to define, which appear in the first and the third equations of \ref{eq:2D-PML-strong-timeD} is the integral of the stress and the strain.
\begin{equation}
\doubleunderline{\Sigma} = \int^t_0 \doubleunderline{\sigma} dt, \doubleunderline{E} = \int^t_0 \doubleunderline{\epsilon} dt
\end{equation} 
\subsection{Displacement-based weak form}
Let us introduce the test function $\underline{v}$ belonging to an admisible space of solution $V$. Premultiplying the first equation of the strong form of the equation of motion within the PML \ref{eq:2D-PML-strong-timeD} by $\underline{v}$ and integrating over the PML domain gives :
\begin{align}
\int_{\Omega_{PML}} \underline{v}.div(\doubleunderline{\sigma}\tilde{F}^e + \doubleunderline{\Sigma}\tilde{F}^p) d\Omega_{PML} & = \int_{\Omega_{PML}} \rho f_m \underline{v}.\underline{\ddot{u}} d\Omega_{PML} + ... \nonumber\\
&  \int_{\Omega_{PML}} \rho \frac{c_s}{b} f_c \underline{v}.\underline{\dot{u}} d\Omega_{PML} +  \int_{\Omega_{PML}} \frac{\mu}{b^2}f_k \underline{v}. \underline{u}  d\Omega_{PML},  \text{In } \Omega_{PML} \times J 
\end{align}
Using Gauss divergence theorem and integration per parts give:
\begin{multline}
\int_{\Gamma_{PML}} \underline{v} .\left(\doubleunderline{\sigma}\tilde{F}^e + \doubleunderline{\Sigma}\tilde{F}^p \right).\underline{n} d\Gamma_{PML}  = \int_{\Omega_{PML}} \rho f_m \underline{\ddot{u}} . \underline{v} d\Omega_{PML} + ... \nonumber\\
\int_{\Omega_{PML}} \rho \frac{c_s}{b} f_c \underline{\dot{u}} . \underline{v} d\Omega_{PML} +  \int_{\Omega_{PML}} \frac{\mu}{b^2}f_k \underline{u} . \underline{v} d\Omega_{PML} +\int_{\Omega_{PML}} \doubleunderline{\tilde{\epsilon}}^e : \doubleunderline{\sigma} d\Omega_{PML} + \int_{\Omega_{PML}} \doubleunderline{\tilde{\epsilon}}^p : \doubleunderline{\Sigma} d\Omega_{PML},  \text{In } \Omega_{PML} \times J 
\label{eq:weak-form-motion}
\end{multline}
$\underline{n}$ is the normal vector to the boundary of the PML $\Gamma_{PML}$. The tensors $\doubleunderline{\tilde{\epsilon}}^p$ and $\doubleunderline{\tilde{\epsilon}}^e$ depend on the attenuation functions as:
\begin{equation}
\begin{cases}
\doubleunderline{\tilde{\epsilon}}^e = \frac{1}{2}\left(\nabla \underline{v} \tilde{F}^e + \tilde{F}^{eT} \nabla \underline{v}^T  \right) \\
\doubleunderline{\tilde{\epsilon}}^p = \frac{1}{2}\left(\nabla \underline{v} \tilde{F}^p + \tilde{F}^{pT} \nabla \underline{v}^T  \right)
\end{cases}
\label{eq:weak-first-eq}
\end{equation}
Using the weak form of the equation of motion within the PML in the temporal domain \ref{eq:weak-form-motion} we can define the mass, damping and stiffness matrices as:
\begin{equation}
m^{e} = \int_{\Omega_e} \rho f_m N_I N_J d\Omega_e I_d
\label{eq:2Dpml-elem-mass}
\end{equation}
\begin{equation}
 c^{e} = \int_{\Omega_e} \rho f_c \frac{c_s}{b} N_I N_J d\Omega_e I_d
 \label{eq:2Dpml-elem-damp}
\end{equation}
\begin{equation}
  k^{e} = \int_{\Omega_e} \frac{\mu}{b^2} f_k N_I N_J d\Omega_e I_d 
  \label{eq:2Dpml-elem-stiff}
\end{equation}
In the above equation $N_I$ is the nodal shape function for node $I$.
Let us introduce a notation for the internal forces term: 
\begin{equation}
p^e = \int_{\Omega_{PML}} \doubleunderline{\tilde{\epsilon}}^e : \doubleunderline{\sigma} d\Omega_{PML} + \int_{\Omega_{PML}} \doubleunderline{\tilde{\epsilon}}^p : \doubleunderline{\Sigma} d\Omega_{PML}
\end{equation}
In order to modify this term we need to make a temporal discretization.
\subsection{Complete discrete form}
Before discribing the complete discrete equations of two-dimensional PML featuring discretization in space ad time, we need to introduce the following notation. In the following, the hat notation above an element will precise that the tensor is written in Voigt notation. For example $\hat{\sigma} = \begin{pmatrix}
\sigma_{11} \\
\sigma_{22} \\
\sigma_{12}
\end{pmatrix}$ 
Thus, using a simple temporal discretization with $dt=t_{n+1} - t_n$ we can rewrite the internal forces at time $t_{n+1}$ as:
\begin{equation}
p_{n+1}^e = \int_{\Omega_e} \tilde{B}^{eT} \hat{\sigma}_{n+1} d\Omega_e + \int_{\Omega_e}\tilde{B}^{pT} \hat{\Sigma}_{n+1} d\Omega_e
\label{eq:intern-forces-discrete}
\end{equation} 
The two matrices $\tilde{B}^{p}$ and $\tilde{B}^{e}$ in \ref{eq:intern-forces-discrete} depend on the nodal shape functions and the attenuation functions. They are expressed in term of their nodal submatrices as:
\begin{equation}
\tilde{B}^e_I = \begin{bmatrix}
\tilde{N}^e_{I1}&0\\0&\tilde{N}^e_{I2}\\\tilde{N}^e_{I2}&\tilde{N}^e_{I1}
\end{bmatrix}, \tilde{B}^p_I = \begin{bmatrix}
\tilde{N}^p_{I1}&0\\0&\tilde{N}^p_{I2}\\\tilde{N}^p_{I2}&\tilde{N}^p_{I1}
\end{bmatrix}
\end{equation} 
with 
\begin{equation}
\tilde{N}^e_{Ii} = \tilde{F}^e_{ji}N_{I,j}, \tilde{N}^p_{Ii} = \tilde{F}^p_{ji}N_{I,j}
\end{equation}
And
\begin{equation}
\tilde{B}^T = \tilde{B}^{eT}+dt \tilde{B}^{pT}
\end{equation}
\begin{equation}
B^\epsilon_I = \begin{bmatrix}
F^\epsilon_{11}N^I_{I1}&F^\epsilon_{21}N^I_{I1}\\
F^\epsilon_{12}N^I_{I2}&F^\epsilon_{22}N^I_{I2}\\
F^\epsilon_{11}N^I_{I2}+F^\epsilon_{12}N^I_{I1}& F^\epsilon_{21}N^I_{I2}+F^\epsilon_{22}N^I_{I1}
\end{bmatrix}
\end{equation}
Some assumptions have to be made to perform time stepping. To evaluate the integral of the stress or strain at the next time step we will assume that:
\begin{equation}
\begin{cases}
\hat{\Sigma}_{n+1} &= \hat{\Sigma}_n + dt \hat{\sigma}_{n+1} \\
\hat{E}_{n+1} &= \hat{E}_n + dt \hat{\epsilon}_{n+1}
\end{cases}
\end{equation}
Thus using this assumption, the internal forces term can be rewrite as:
\begin{equation}
p_{n+1}^e = \int_{\Omega_e} \tilde{B}^{T} \hat{\sigma}_{n+1} d\Omega_e + \int_{\Omega_e}\tilde{B}^{pT} \hat{\Sigma}_{n} d\Omega_e
\end{equation}
where
\begin{equation}
\tilde{B}^{T} = \tilde{B}^{eT} + dt \tilde{B}^{pT}
\end{equation}
An additional assumption has to be made to evaluate the derivative of the strain:
\begin{equation}
\dot{\epsilon}_{n+1} = \frac{\epsilon_{n+1}-\epsilon_n}{dt}
\end{equation} 
This correspond to the first order approxiation due to Taylor's theorem.
Using this assumption in the third equation of \ref{eq:2D-PML-strong-timeD} the expression of the strain for the next time step can be obtained.
\begin{equation}
\hat{\epsilon}_{n+1} = \frac{1}{dt}\left(B^\epsilon \dot{U}_{n+1} + B^Q U_{n+1} + \frac{1}{dt} \hat{F}^\epsilon \hat{\epsilon}_n - \hat{F}^Q \hat{E}_n\right)
\label{eq:strain-n+1}
\end{equation} 
The matrices $B^\epsilon$, $B^Q$, $\hat{F}^\epsilon$ and $\hat{F}^Q$ depend on the attenuation and the nodal shape functions. The are defined by their nodal submatrices as:
\begin{equation}
B^\epsilon_I = \begin{bmatrix}
F^\epsilon_{11}N^I_{I1}&F^\epsilon_{21}N^I_{I1}\\
F^\epsilon_{12}N^I_{I2}&F^\epsilon_{22}N^I_{I2}\\
F^\epsilon_{11}N^I_{I2}+F^\epsilon_{12}N^I_{I1}& F^\epsilon_{21}N^I_{I2}+F^\epsilon_{22}N^I_{I1}
\end{bmatrix}
\end{equation}
with
\begin{equation}
N^I_{Ii} = F^I_{ij}N_{I,j}
\end{equation}
\begin{equation}
F^I = \left[ F^p+\frac{F^e}{dt} \right]^{-1}, F^\epsilon = F^eF^I, F^Q=F^pF^I
\end{equation}
and
\begin{equation}
\hat{F}^\epsilon_I = \begin{bmatrix}
(F^\epsilon_{11})^2&(F^\epsilon_{21})^2& F^\epsilon_{11}F^\epsilon_{21}\\
(F^\epsilon_{12})^2&(F^\epsilon_{22})^2&F^\epsilon_{12}F^\epsilon_{22}\\
2F^\epsilon_{11}F^\epsilon_{12}&2 F^\epsilon_{21}F^\epsilon_{22}& F^\epsilon_{11}F^\epsilon_{22}+F^\epsilon_{12}F^\epsilon_{21}
\end{bmatrix}
\end{equation}
To obtain the relations for $B^Q$ and $\hat{F}^Q$ the only thing to replace is the superscript $\epsilon$ by $Q$. 
\par The stress $\hat{sigma}_{n+}$ is computed using the second equation of \ref{eq:2D-PML-strong-timeD} involving the elastic constitutive tensor $C$. Thus the internal forces term can be expressed in function of the strain and not the stress.
\begin{equation}
p_{n+1}^e = \int_{\Omega_e} \tilde{B}^{T} D \hat{\epsilon}_{n+1} d\Omega_e + \int_{\Omega_e}\tilde{B}^{pT} \hat{\Sigma}_{n} d\Omega_e
\end{equation} 
and using the relation \ref{eq:strain-n+1} this expression can be rewrite as:
\begin{align}
p_{n+1}^e &= \int_{\Omega_e} \tilde{B}^{T} D \left[\frac{1}{dt}\left(B^\epsilon \dot{U}_{n+1} + B^Q U_{n+1} + \frac{1}{dt} \hat{F}^\epsilon \hat{\epsilon}_n - \hat{F}^Q \hat{E}_n\right)  \right] d\Omega_e + \int_{\Omega_e}\tilde{B}^{pT} \hat{\Sigma}_{n} d\Omega_e \\
&= \tilde{c}^e \dot{U}^e_{n+1} + \tilde{k}^e U^e_{n+1} + P(\hat{\epsilon}_n,\hat{E}_n,\hat{\Sigma}_n)
\end{align} 
where 
\begin{equation}
P(\hat{\epsilon}_n,\hat{E}_n,\hat{\Sigma}_n) = \int_{\Omega_e} \tilde{B}^T \frac{D}{dt} \left[\frac{1}{dt}\hat{F}^{\epsilon} \hat{\epsilon} - \hat{F}^Q \hat{E}_n \right] + \tilde{B}^p \hat{\Sigma}_n d\Omega_e
\label{eq:pseudo-intern}
\end{equation}
and 
\begin{equation}
\tilde{c}^{e} = \frac{1}{dt} \int_{\Omega_e} \tilde{B}^T D B^\epsilon d\Omega_e
\label{eq:2Dpml-elem-effdamp}
\end{equation}
\begin{equation}
\tilde{k}^{e} = \frac{1}{dt} \int_{\Omega_e} \tilde{B}^T D B^Q d\Omega_e
\label{eq:2Dpml-elem-effstiff}
\end{equation}
Under the plane-strain assumption the material constitutive matrix is expressed as:
\begin{equation}
\begin{pmatrix}
K+\frac{4}{3}\mu_L& K-\frac{2}{3}\mu_L&0\\
K-\frac{2}{3}\mu_L&K+\frac{4}{3}\mu_L&0\\
0&0&\mu_L
\end{pmatrix}
\end{equation}
Therefore using the above equation we can rewrite \ref{eq:weak-first-eq} as:
\begin{equation}
M\ddot{U}_{n+1} + \left(C+\tilde{C}\right)\dot{U}_{n+1} + \left(K+\tilde{K}\right)U_{n+1} + P(\hat{\epsilon}_n,\hat{E}_n,\hat{\Sigma}_n) = F_{ext}
\label{eq:2Dpml-discrete-motion}
\end{equation}
Where $M$,$C$ and $K$ are respectively the mass, damping and stiffness matrices resulting from the assembly of their element matrices \ref{eq:2Dpml-elem-mass}, \ref{eq:2Dpml-elem-damp} and \ref{eq:2Dpml-elem-stiff}. The matrices $\tilde{K}$ and $\tilde{C}$ derive from the assembly procedure of their element matrices \ref{eq:2Dpml-elem-effstiff} and \ref{eq:2Dpml-elem-effdamp}. $P(\hat{\epsilon}_n,\hat{E}_n,\hat{\Sigma}_n)$ is given by the equation \ref{eq:pseudo-intern} and is known at the beginning of the time step since it depends only on components of the previous time. 
\subsection{Time integration scheme}
The complete discrete equations obtained in the previous part acn be integrated using the Newmark-$\beta$ implicit scheme. The classical Newmark approximation formulas are expressed in acceleration form:
\begin{equation}
\begin{cases}
U_{n+1} = U_{n,p} + \beta dt^2 \ddot{U}_{n+1} \\
\dot{U}_{n+1} = \dot{U}_{n,p} + \gamma dt \ddot{U}_{n+1}
\end{cases}
\label{eq:newmark1}
\end{equation}
where the predictors $U_{n,p}$ and $\dot{U}_{n,p}$ have the form:
\begin{equation}
\begin{cases}
U_{n,p} = U_n + dt \dot{U}_n + dt^2 \left(\frac{1}{2} -\beta  \right)\ddot{U}_n \\
\dot{U_{n,p}} = \dot{U}_n + dt (1-\gamma)\ddot{U}_n
\end{cases}
\label{eq:newmark2}
\end{equation}
Substituing the equations \ref{eq:newmark1} and \ref{eq:newmark2} into the discrete form of the equation of motion \ref{eq:2Dpml-discrete-motion} the acceleration at time $t_{n+1}$ can be obtained.
\begin{equation}
\tilde{M}\ddot{U}_{n+1} = F_{ext} - \left(C+\tilde{C}\right)\dot{U}_{n,p} - \left(K+\tilde{K}\right)U_{n,p} - P(\hat{\epsilon}_n,\hat{E}_n,\hat{\Sigma}_n)
\end{equation}
where 
\begin{equation}
\tilde{M} = M + \gamma dt \left(C+\tilde{C}\right) + \beta dt^2 \left(K+\tilde{K}\right)
\end{equation}
This matrix need to be inverted, this will be done before beginning the time stepping since all the matrices constituing it remain constant through the time stepping. 









 






\newpage
\section{Methods and results}
With the implementation of the PML and using the algorithm \ref{algo}, we can now proceed to the test of the one dimensional PML. To evaluate the performance, we will focus on the reflection of the wave from the interface and the wave coming back from the PML. The other parameter to be evaluate will be the computational cost in term of time. In fact the objective of such analysis is to find the best compromise between accuracy, minimising the reflection wave and the computational. Therefore we will be able to highlight the parameters to have the best result possible.
\subsection{Inputs and parameters}
In order to test the PML, we choose to place ourselves in the context of the problem described by the figure \ref{fig:sch_pml}. A summary of the parameters used for the simulation can be found in the following table.
\begin{table}[H]
    \centering
    \begin{tabular}{c|c}
         $L_m$ & 50  \\
         $L_p$ & 20 \\
         $L_e$ & 1 \\
         $S$ & 1 \\
         $c_s$ & 50 \\
         $\rho$ & 1700\\
         E & $10e^5$ \\
         h & $0.8 h_{CFL}$ \\
         T & $500$ \\
    \end{tabular}
    \caption{Parameters for the PML and the medium}
    \label{tab:param}
\end{table}
In the table \ref{tab:param}, $T$ stands for the end time. The time step $h$ must be chosen carefully. This parameter is very important since we seek a balance between computational cost and accuracy. Of course increasing this parameter leads to a lower computational cost but also to a smaller accuracy. Another property determined by this parameter is also the stability of the numerical scheme. The method implemented in our case uses a GC coupled interface in order to be able to perform an explicit scheme on the medium while conserving an implicit one on the PML. Since explicit Newmark scheme is conditionally stable we need to ensure that the time step is smaller than the critical time step determine by the formula:
\begin{equation}
    h_{CFL} = \frac{L_e}{\sqrt{\frac{E}{\rho}}}
\end{equation}
A simple numerical application gives us in our case $h_{CFL} = 0.0412$. We ensured that the time step is below this critical time step.
The parameters for the Newmark-$\beta$ scheme are $\gamma=0.5$ and $\beta=0.25$.
The last parameters to define are the ones for the attenuation functions. The order of the polynomials will be $m=2$. The coefficients for the attenuation function for propagating wave is the most important to define due to its importance for the damping of the wave. If we assume that $\frac{c_s}{L_p}=1$, the coefficient for this function and in the case of a propagating P-wave in a one dimensional medium can be expressed by the following formula \cite{Kucukcoban2}:
\begin{equation}
    \alpha_p = \frac{(m+1)v_p}{2 L_{p}}log\left( \frac{1}{R} \right)
\end{equation}
where $v_p$ is the velocity of the P-waves and $R$ is the reflection coefficient from the outer PML boundary. This coefficient shows the amplitude reduction of the P wave entering into the PML. This parameter has to be taken into account when we will evaluate the performance of the method. The value for the parameters in the expression of the coefficient $\alpha_p$ has, in our example, the following values:
\begin{table}[H]
    \centering
    \begin{tabular}{c|c}
         $v_p$ & $76.7$  \\
         $R$ & $10^{-3}$ \\
         $\alpha_e$ & 1 
    \end{tabular}
    \caption{Parameters for the attenuation functions}
    \label{tab:param_attenuations}
\end{table}
As shown in the table \ref{tab:param_attenuations}, the coefficient of attenuation for evanescent wave is simply $0$. Since this function does not have a large impact on the damping of the wave entering the PML: $f^e(x)=0 \forall x$. 

\subsection{Ricker wave}
In this numerical simulation, we will analyse the behavior of the PML against the propagation of non harmonic waves. A Ricker wave will be imposed at the free end of the physical medium in the form of a force. This external force has the following formula:
\begin{equation}
    F_{ext} = A\left( 2 \pi^2\frac{(t-t_s)^2}{t^2_p}-1\right) exp\left( -\pi^2\frac{(t-t_s)^2}{t^2_p}\right)
\end{equation}
The Ricker wave is defined by 3 parameters: The fundamental period $t_p$, the time shift $t_s$ and the amplitude $A$. These parameters take the values:
\begin{table}[H]
    \centering
    \begin{tabular}{c|c}
         $t_p$ & 0.5s \\
         $t_s $& 1s \\
         $A$ & 1
    \end{tabular}
    \caption{Values for the parameters of the Ricker Wave}
    \label{tab:param_Ricker}
\end{table}
Using this parameters the Ricker wave in the time domain has the following shape:
\begin{figure}[H]
    \centering
    \includegraphics[scale=0.5]{images/ricker.png}
    \caption{Ricker wave in the time domain}
    \label{fig:ricker}
\end{figure}
\subsection{Results}
The measurement of the displacement and the velocity at the center of the physical medium gives a good insight of the propagation of the wave and of its possible reflection. Using the above parameters for the medium, the PML and the input wave, the following graphs are obtained:
\begin{figure}[H]
\centering
\begin{minipage}{.5\textwidth}
  \centering
  \includegraphics[width=.9\linewidth]{images/disp_med.png}
\end{minipage}%
\begin{minipage}{.5\textwidth}
  \centering
  \includegraphics[width=.9\linewidth]{images/vel_med.png}
\end{minipage}
\caption{Longitudinal displacement (right) and the velocity (left) at $x=25m$ as a function of time}
\label{fig:disp_vel_improved}
\end{figure}

The figure \ref{fig:disp_vel_improved} shows the displacement and the velocity through the time at the node placed at $x=25m$. 
Looking at the figures \ref{fig:disp_vel_improved} we can see that the PML reduced drastically the reflection of the incident wave at the interface. Therefore the PML seems accurate since we can barely see the reflection of the wave at the interface. \\

\subsubsection{Reflection of the wave}
In this part, we will focus on the reflection of the wave in function of different parameters. To be more precise, we will separate the reflection of the wake into two kinds: the reflection due to the interface and the reflected wave bouncing at the extremity of the PML and coming back into the medium. This study will focus on the effect of the parameters for the PML such as the length $L_p$ and the attenuation coefficient $\alpha_p$. The length of the elements $L_e$ will also be tested and the last parameter tested will be the time step $k$.\\ 
First of all let us focus on the effect of the attenuation coefficient $\alpha_p$.
\begin{figure}[H]
\centering
\begin{minipage}{.5\textwidth}
  \centering
  \includegraphics[width=.8\linewidth]{images/reflected_alphap.png}
\end{minipage}%
\begin{minipage}{.5\textwidth}
  \centering
  \includegraphics[width=.8\linewidth]{images/bouncing_alphap.png}
\end{minipage}
\caption{Percentage of reflected wave due to the interface (left) and coming back from the extremity of PML (right) in function of $\alpha_p$}
\label{fig:refl_alphap}
\end{figure}
As shown on the figure \ref{fig:refl_alphap}, the attenuation coefficient has no effect on the reflection of the wave at the interface: there is exactly no reflection due to the interface.  The wave is damped in the PML in function of this coefficient as shown on the right figure. Indeed if $\alpha_p = 0$ There is no damping of the wave whereas if $\alpha_p = 10$ the wave is totally attenuated by the PML.\\
Let us now investigate the effect of the length of the PML. Since it has any effects on the interface we will focus only on the percentage of the wave coming back from the absorbing layer. 
\begin{figure}[H]
    \centering
    \includegraphics[scale=0.5]{images/bouncing_Lp.png}
    \caption{Percentage of reflected wave in function of $L_p$}
    \label{fig:refl_Lp}
\end{figure}
As we can observe on the figure \ref{fig:refl_Lp}, the percentage of the wave reflected at the fixed extremity of the PML and coming back in the physical medium decreases when the length of the PML increases. \\
% Length of element
Let us now focus on the length of the element $L_e$ for $20$ elements in the PML. Since the length of the PML and its coefficients of attenuation permits the complete attenuation of the incident wave, we will focus only on the reflection due to the interface.
\begin{figure}[H]
    \centering
    \includegraphics[scale=0.5]{images/reflected_Le.png}
    \caption{Percentage of reflected wave at the interface in function of $L_e$}
    \label{fig:refl_Le}
\end{figure}
As we can see on the figure \ref{fig:refl_Le}, the slope of the percentage of the reflected wave can be evaluated at $L_e^1$.










\newpage
\section{Stability}
After evaluating numerically the overall efficiency of the method for the one dimensional case, the theoretical and numerical stability will be studied. In fact, the theoretical stability does not lead to the stability of the numerical scheme. Therefore, both stabilities will be studied separately
\subsection{Theoretical stability}
\subsubsection{Well-posedness}
The definition of the well-posedness is that a problem is said to be well-posed if it has a solution, that should be unique and should depend continuously on the data of the problem. This last requirement states that perturbations such as errors in measurement should not affect the solution in a too large proportion. 


\subsection{Stability of the numerical method}
In order to prove the stability of the temporal scheme, we can recall the following result concerning integration methods \cite{Geradin}: An integration scheme is said to be stable if there exists an integration step $h_0 > 0$ so that for any $h \in [0, h_0]$, a finite variation of the state vector at time $t^n$ induces only a non-increasing variation of the state vector $q^{n+1}$ calculated at a subsequent instant $t^{n+1}$.\\
In our case we will study the stability of the scheme with the temporal discretization presented in part \ref{part:describ}. This reduces to the analysis of the stability properties of the Newmark-$\beta$ scheme used in the algorithm \ref{algo} for the PML uniquely. We will focus our attention of the stability of the temporal scheme applied on only one element. We can prove the stability on one element the result extends to the other element \cite{Belytschko}. First of all we need to go back to the equations of motion and define the state vector: $Q^n = [\dot{u}^n, u^n, H^n]^T$.
The objective is to define the matrices $A$ and $B$ such that:
\begin{equation}
    A(h) Q^{n+1} + B(h) Q^{n} = 0
    \label{eq:matrix_form}
\end{equation}
which is in fact the matrix form of the equations of motion. Since each component of the state vector have the size $2 \times 1$, the matrices $A$ and $B$ are $6 \times 6$. This formulation comes from the fact that the we used have only one bar element composed of two points.    
Let us first recall the equation of motion at time $t^n$ and $t^{n+1}$:
\begin{equation}
    M  \Ddot{u}^n = -C \Dot{u}^n - K u^n +p_{int}^n 
\end{equation}
\begin{equation}
    M \Ddot{u}^{n+1} = -C \Dot{u}^{n+1} - K u^{n+1} +p_{int}^{n+1} 
\end{equation}
Using the recurrence relationship described in the algorithm \ref{algo} from the Newmark scheme, we obtain:
\begin{equation}
    M \dot{u}^{n+1} = M\Dot{u}^{n} + h(1-\gamma)\left[ -C \Dot{u}^n - K u^n +p_{int}^n   \right] + \gamma h \left[ -C \Dot{u}^{n+1} - K u^{n+1} +p_{int}^{n+1} \right]
    \label{eq:rec1}
\end{equation}

\begin{equation}
    M u^{n+1} = M u^{n} + h M \dot{u}^n + h^2(\frac{1}{2}-\beta)\left[ -C \Dot{u}^n - K u^n +p_{int}^n   \right] + \beta h^2 \left[ -C \Dot{u}^{n+1} - K u^{n+1} +p_{int}^{n+1} \right]
    \label{eq:rec2}
\end{equation}
With $\beta$ and $\gamma$ the parameters of the Newmark scheme. We also have a recurrence relation for $H^{n+1}$. 
\begin{equation}
    H^{n+1} = H^n + h \epsilon^{n+1}
    \label{eq:rec3}
\end{equation}
With $\epsilon^{n+1}$ the strain at time $t^{n+1}$ which can be expressed in function of $u^{n+1}$ and $H^n$. The following formulation is element wise, since we consider the stability on only one element this formulation suffices:
\begin{equation}
    \epsilon^{{n+1}^e} = \frac{[B]\{u^{{n+1}^e}\} - \frac{c_s}{L_p} f^p_x H^n}{\alpha}
    \label{eq:eps}
\end{equation}
with $\alpha = (1+f^e_x)+\frac{c_s}{L_p} f^p_x $.\\ 
Using the equations \ref{eq:rec1}, \ref{eq:rec2} and \ref{eq:rec3}, we seek the definition of the matrix form \ref{eq:matrix_form}. We will define the matrices per blocs of size $2 \times 2$. In the following development the mass $M$, damping $C$ and stiffness $K$ matrices are using the formulas \ref{eq:me}, \ref{eq:ce} and \ref{eq:Ke} for one element.  \\
First of all, let us multiply by the left \ref{eq:rec1} and \ref{eq:rec2} by $M^{-1}$ and put all the terms to the left side:

\begin{multline}
    \dot{u}^{n+1}\left[I + \gamma h M^{-1}C\right] + u^{n+1}\left[\gamma h M^{-1}K \right] -\gamma h M^{-1} p_{int}^{n+1}  
    + \dot{u}^n\left[-I + h(1-\gamma)M^{-1}C   \right] \\+ u^n\left[ h(1-\gamma) M^{-1}K \right] - h(1-\gamma)M^{-1} p_{int}^n = 0
\end{multline}
\begin{multline}
    \dot{u}^{n+1}\left[\beta h^2 M^{-1} C \right] +  u^{n+1}\left[I+\beta h^2 M^{-1} K  \right] - \beta h^2 M^{-1} p_{int}^{n+1} + \dot{u}^n \left[ -h I + h^2(\frac{1}{2}-\beta) M^{-1} C \right] \\
    + u^n\left[-I + h^2(\frac{1}{2}-\beta) M^{-1} K \right] - h^2(\frac{1}{2}-\beta) M^{-1}p^n_{int}=0
\end{multline}
The only problem remaining before constructing the matrices $A(h)$ and $B(h)$ is how to handle the term of the internal forces $p_{int}$.
Let us recall its expression first:
\begin{equation}
    p_{int}^n = \int^{1}_{-1}[B]^T\frac{c_s}{L_p}\frac{f^p(\xi)}{\alpha(\xi)} H^n S  \frac{L_e}{2} d\xi
\end{equation}
Since we assumed that $f^p$ and $f^e$ are constant and using the quadrature to evaluate the integral with two Gauss points with weights $w=1$  we obtain

\begin{equation}
    p_{int}^n = \frac{c_s}{L_p} E S \begin{bmatrix} \frac{2}{L_e} \\ \frac{-2}{L_e} \end{bmatrix} \begin{bmatrix} 1 \times \frac{H_{n,k=1} f^p_{k=1}}{\alpha_{k=1}}&  1 \times \frac{H_{n,k=2} f^p_{k=2}}{\alpha_{k=2}}  \\ \end{bmatrix} = \frac{c_s}{L_p} E S \frac{f^p}{\alpha} \begin{bmatrix}
        1 &1\\
        -1 & -1
    \end{bmatrix} \begin{bmatrix}H_{n,k=1} \\ H_{n,k=2}\end{bmatrix}
\end{equation} 
where $k=1$ corresponds to the first Gauss point and $k=2$ to the second. We also have:
\begin{equation}
    A_p = \frac{c_s E S}{L_p}\frac{f^p_x}{\alpha} \begin{bmatrix}
        1 &1\\
        -1 & -1
    \end{bmatrix}
\end{equation}
For the last row of the matrices we need to focus on the equation:
\begin{equation}
    H^{n+1} = H^n + h \epsilon^{n+1}
\end{equation}
with $\epsilon^{n+1}$ given by the equation \ref{eq:eps} such that we obtain:
\begin{equation}
    H^{n+1}+H^n\left[-I + h \frac{c_s}{L_p} \frac{f^p_x}{\alpha} \right] - \frac{h}{\alpha} \left[\overline{B} \right]u^{n+1} 
\end{equation}
\begin{equation}
    A(h) = \begin{bmatrix} I+\gamma h M^{-1} C & \gamma h M^{-1} K & -\gamma h M^{-1} A_p \\
    \beta h^2 M^{-1} C & I+\beta h^2 M^{-1} K & -\beta h^2 M^{-1} A_p\\
    0 & -\frac{h}{\alpha} \overline{B} & I
    \end{bmatrix}
    \label{eq:A(h)}
\end{equation}
where 
\begin{equation}
    \overline{B} = \frac{1}{L_e} \begin{bmatrix}
        -1 & 1\\
        -1 & 1
    \end{bmatrix}
\end{equation}
We expressed in \ref{eq:A(h)} all the terms depending on time $t^{n+1}$, let us now express the remaining terms depending on time $t^n$:
\begin{equation}
    B(h) = \begin{bmatrix} -I+(1-\gamma) h M^{-1} C & (1-\gamma) h M^{-1} K & -(1-\gamma) h M^{-1} A_p \\
    -h I + (\frac{1}{2}-\beta) h^2 M^{-1} C & -I+(\frac{1}{2}-\beta) h^2 M^{-1} K & -(\frac{1}{2}-\beta) h^2 M^{-1} A_p\\
    0 & 0 & -I+\frac{c_s}{L_p}\frac{f^p_x}{\alpha} I
    \end{bmatrix}
    \label{eq:B(h)}
\end{equation}
Using the equation \ref{eq:A(h)} and \ref{eq:B(h)}, we can define the amplification matrix associated with the integration operator $H(h) = A(h)^{-1} B(h)$.\\
As shown in \cite{Geradin}, the stability of the integration method is ensured if the eigenvalues of the amplification matrix $H(h)$ are contained in the unit circle i.e.\ the moduli of the eigenvalues are lower than unity.\\ To prove the stability of our temporal scheme let us consider the simplest case where $f^e_x = 0$ everywhere and $f^p_x$ is a constant equals to $10$. Let us also consider the case of the unconditionally stable Newmark scheme ie $\beta = 0.25$ and $\gamma = 0.5$. We will consider several values of time steps and for each of them we will plot the eigenvalues of the amplification matrix in the complex plane. For $h$ taking its value from $0.001$ to $10$ by step of $0.001$ we obtain the following figure:
\begin{figure}[H]
    \centering
    \includegraphics[scale=0.6]{images/radius.png}
    \caption{Eigenvalues of the amplification matrix in the complex plane}
    \label{fig:eigenval_amp}
\end{figure}
As we can observe on figure \ref{fig:eigenval_amp}, the eigenvalues of the amplification matrix $H(h)$ for different values of $h$ remain in the unit circle. When $h$ increases the eigenvalues are positioned at the left of the unit circle. For small values of $h$ the eigenvalues are at the right. For more complex case including those where the attenuation function are not assumed constant this result holds. This result ensures that the temporal integration scheme , we developed is stable in time for the one dimensional PML. This result shows that the integration scheme derived from the equation of the PML is stable in time. \\
In order to evaluate the properties of a numerical scheme in terms of stability, an interesting method is to look at the spectral radius, the periodicity error and the numerical damping. The spectral radius of the method is given by the modulus of the highest eigenvalue:
\begin{equation}
    R(A) = \max_i(|\lambda_i|)
\end{equation}
We will express this parameter in function of $\Omega = \omega h$ where $\omega = \sqrt{\frac{k}{m}}$ and $k=\frac{E S}{L_p}$. $m$ is the mass of the bar (in our case just an element). The asymptotic value of the spectral radius for $\omega h \rightarrow \infty$ gives an information about the stability of the method over the entire frequency domain.
\begin{figure}[H]
    \centering
    \includegraphics[scale=0.6]{images/spectral_rad.png}
    \caption{Spectral radius in function of $h\omega$}
    \label{fig:spectral_rad}
\end{figure}
As shown on the figure \ref{fig:spectral_rad}, the spectral radius remains under the value $1$ and its asymptotic value tends to $1$ as  $\omega h \rightarrow \infty$. This demonstrates that the numerical scheme is stable over the entire frequency domain. 
Let us now focus on the periodicity error which results from the comparison between the numerical and the exact frequencies obtained for the one degree-of-freedom oscillator:
\begin{equation}
    \frac{\Delta T}{T} = \frac{\omega h}{ \phi} - 1 \hspace{2cm}\textnormal{where} \hspace{1cm} \phi=tan^{-1}\left(\frac{\operatorname{Im} \lambda_1}{\operatorname{Re} \lambda_1}  \right)
    \label{eq:period_err}
\end{equation}
Therefore this is a measurement of the frequency distortion for each eigenfrequency in the model. 

\begin{figure}[H]
    \centering
    \includegraphics[scale=0.6]{images/relative_periodicity_error.png}
    \caption{relative periodicity error in function of $h\omega$}
    \label{fig:relative_per_err}
\end{figure}
As observed in \ref{fig:relative_per_err}, the relative periodicity error increases with $h \omega$ looking back at the equation \ref{eq:period_err} and knowing that the modulus of each eigenvalue is smaller or equal to $1$, the result obtained here about the relative periodicity error is the one expected. Since this result is difficult to interpret a more complete analysis will be done in the next report. Indeed some questions remain unanswered such as why the distortion at $h=0$ is different than $0$ or other questions about the numerical damping ratio we will present now. \\
The last parameter to be investigate is the numerical damping ratio. It compare the time decay coefficient of the real part of the solution to the numerical frequency.
\begin{equation}
    \xi = -\frac{log(|\lambda_1|)}{\phi}
    \label{eq:num_damp}
\end{equation}
It measures the percentage of critical damping introduced in the system by the integration operator. In fact it is expected to damped out the high frequencies introduced by the time integration operator.   
\begin{figure}[H]
    \centering
    \includegraphics[scale=0.6]{images/Numerical_damping_ratio.png}
    \caption{Numerical damping ratio in function of $h\omega$}
    \label{fig:num_damp_ratio}
\end{figure}
In order to compare the results obtained on the Newmark-$\beta$ unconditionally stable scheme, let us make the same analysis of stability on the Newmark explicit temporal integration scheme for the PML with $\beta=0$ and $\alpha=0.5$ (central differences). 
\begin{figure}[H]
\centering
\begin{minipage}[b]{0.475\textwidth}
    \centering
    \includegraphics[width=\textwidth]{images/pml_expl_eig.png}
    \caption[Network2]%
    {{\small Eigenvalues in the complex plane}}    
    \label{fig:med_eig}
\end{minipage}
\hfill
\begin{minipage}[b]{0.475\textwidth}  
    \centering 
    \includegraphics[width=\textwidth]{images/spectre_expl.png}
    \caption[]%
    {{\small Spectral radius in function of $h\omega$}}    
    \label{fig:med_spectre}
\end{minipage}
\vskip\baselineskip
\begin{minipage}[b]{0.475\textwidth}   
    \centering 
    \includegraphics[width=\textwidth]{images/pml_exp_per.png}
    \caption[]%
    {{\small relative periodicity error in function of $h\omega$}}    
    \label{fig:med_relat_per}
\end{minipage}
\quad
\begin{minipage}[b]{0.475\textwidth}   
    \centering 
    \includegraphics[width=\textwidth]{images/pml_expl_damp.png}
    \caption[]%
    {{\small Numerical damping ratio in function of $h\omega$}}    
    \label{fig:med_damp}
\end{minipage}
\end{figure}
As expected for a conditionally stable scheme, passed a certain value of $h \omega$ the scheme becomes unstable. This limit of stability is evaluated $h \omega = 2$ corresponding to the CFL condition.\\
We can also look at the behaviour of the Newmark explicit time integrator on the physical medium. In this case the equation of motion are the same as in a classic elastodynamic problem. In the following equation the state vector $q$ will be defined by $q^n=[\dot{u}^n, u^n]^T$. THe equation of motion at time $t^n$ and $t^{n+1}$ are:
\begin{equation}
\begin{aligned}
    M  \Ddot{u}^n &= -C \Dot{u}^n - K u^n +p^n  \\
    M \Ddot{u}^{n+1} &= -C \Dot{u}^{n+1} - K u^{n+1} +p^{n+1} 
\end{aligned}
\end{equation}
And taking into account the recurrence relationship given by the Newmark method:
\begin{equation}
\begin{aligned}
    M \dot{u}^{n+1} &= M\Dot{u}^{n} + h(1-\gamma)\left[ -C \Dot{u}^n - K u^n +p^n   \right] + \gamma h \left[ -C \Dot{u}^{n+1} - K u^{n+1} +p^{n+1} \right] \\
    M u^{n+1} &= M u^{n} + h M \dot{u}^n + h^2(\frac{1}{2}-\beta)\left[ -C \Dot{u}^n - K u^n +p^n   \right] + \beta h^2 \left[ -C \Dot{u}^{n+1} - K u^{n+1} +p^{n+1} \right]
\end{aligned}
\label{eq:med_rec}
\end{equation}
Therefore we can define the matrix form of this recurrence:
\begin{equation}
    q^{n+1} = A(h) q^n + g^{n+1}(h)
\end{equation}
The component of this relation can be defined using the equations \ref{eq:med_rec}. We obtain:
\begin{equation}
    A(h) = \begin{bmatrix}M+\gamma h C & \gamma h K \\ \beta h^2 C & M+ \beta h^2 K   \end{bmatrix}^{-1} \begin{bmatrix}(1-\gamma)hC-M& (1-\gamma)hK \\ (\frac{1}{2}-\beta)h^2C-hM& (\frac{1}{2}-\beta)h^2K-M \end{bmatrix} 
\end{equation}
and where
\begin{equation}
    g^{n+1}=\begin{bmatrix}M+\gamma h C & \gamma h K \\ \beta h^2 C & M+ \beta h^2 K   \end{bmatrix}^{-1} \begin{bmatrix}(1-\gamma)h p^n+\gamma h p^{n+1} \\ (\frac{1}{2}-\beta)h^2p^n+\beta h^2 p^{n+1}  \end{bmatrix}
\end{equation}
The matrix $A(h)$ is the matrix of amplification, its eigenvalues represents the normal vibration modes of the one dimensional element considered. Since the physical medium was assumed to have no damping $C=0$. Let us now expend the solution with respect to the eigenmodes of the structure. The equations of motion can be reduced to uncoupled normal equations:
\begin{equation}
    \Ddot{\eta}^n = - \omega \eta^n+\phi^n
\end{equation}
where $\phi^n$ is the participation factor at the excitation at time $t^n$. Thus using this system of uncoupled equations we can redifine the amplification matrix by:
\begin{equation}
    A(h) = \begin{bmatrix} 1& \gamma h \omega^2\\ 0&1+\beta h^2 \omega ^2\end{bmatrix}^{-1} \begin{bmatrix} 1& -(1-\gamma) h \omega^2\\ h&1-(\frac{1}{2}-\beta) h^2 \omega ^2\end{bmatrix}
\end{equation}
Using this definition of the amplification matrix, we can do the same analysis for the physical medium than the one done of the implicit Newmark scheme on the PML.
\begin{figure}[H]
\centering
\begin{minipage}[b]{0.475\textwidth}
    \centering
    \includegraphics[width=\textwidth]{images/med_rad.png}
    \caption[Network2]%
    {{\small Eigenvalues in the complex plane}}    
    \label{fig:med_eig}
\end{minipage}
\hfill
\begin{minipage}[b]{0.475\textwidth}  
    \centering 
    \includegraphics[width=\textwidth]{images/med_spectre.png}
    \caption[]%
    {{\small Spectral radius in function of $h\omega$}}    
    \label{fig:med_spectre}
\end{minipage}
\vskip\baselineskip
\begin{minipage}[b]{0.475\textwidth}   
    \centering 
    \includegraphics[width=\textwidth]{images/med_period.png}
    \caption[]%
    {{\small relative periodicity error in function of $h\omega$}}    
    \label{fig:med_relat_per}
\end{minipage}
\quad
\begin{minipage}[b]{0.475\textwidth}   
    \centering 
    \includegraphics[width=\textwidth]{images/med_damp.png}
    \caption[]%
    {{\small Numerical damping ratio in function of $h\omega$}}    
    \label{fig:med_damp}
\end{minipage}
\end{figure}
We can observe on the figure \ref{fig:med_eig} that for a certain value of $h$ the eigenvalues seems to exit the unit circle. This is the proof that at a certain point the scheme becomes unstable and amplified drastically even small perturbations. This change from stable to unstable occurs when $h\omega > 2$ as we can see on the other figures






\newpage
\begin{frame}{Conclusion}
\begin{itemize}
\item Overview of the analysis of stability for integration method.
\item Stability of 1D and 2D perfectly matched layer (for implicit time integration method).
\item Properties of attenuation and delay of unstability for 1D PML.
\item \underline{Further work :}
\begin{itemize}
\item Same analysis of relative periodicity error and numerical damping for 2D PML.
\item Implementation of 2D Perfectly matched layer within Akantu. 
\end{itemize} 
\end{itemize}
\end{frame}


\newpage
%biblio
% Bibliographie
\bibliographystyle{plain}
\bibliography{biblio}
\end{document}
