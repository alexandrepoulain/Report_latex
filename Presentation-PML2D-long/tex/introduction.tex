\section{Introduction}
\subsection{Absorbing boundaries}
\begin{frame}{Absorbing boundaries}
    \begin{itemize}[<+->]
        \item Common practice to solve numerically wave propagation on unbounded domains.
        \item Important topic for many research and engineering applications.
        \item Simulation of earthquake ground motion, for soil-structure, geophysical, subsurface sensing, waveguides problems.
        \pause
        \vspace{0.7cm}
        \item 2 kinds of method:
        \begin{itemize}[<+->]
        \pause
            \item \underline{Absorbing boundary conditions :} specific conditions at the model boundaries.
        \pause
            \item \underline{Absorbing boundary layers (ABL) :} layer surrounding the domain of interest.
        \end{itemize}    
    \end{itemize}
\end{frame}

\subsection{Perfectly matched layers}
\begin{frame}{Perfectly matched layers}
\pause
    \begin{itemize}
        \item \underline{split-field PML:} 
        \begin{itemize}
        \pause
            \item Introduced by Bérenger in the context of electromagnetics.
            \item Extended by Hastings et al to elastodynamics.
            \item Use on complex-valued coordinate stretching.
            \item To avoid convolutional operations in the time domain.
            \item Field splitting: partition of the variables into two components parallel and perpendicular to the truncation boundary.
        \end{itemize}
    \item \underline{unsplit-field PML:} 
        \begin{itemize}
        \pause
            \item Introduced by Wang in the context of elastodynamics (CPML) $\rightarrow$ \textcolor{red}{Complexity} 
            \pause
            \item Basu and Chopra: unsplit-field PML for time-harmonic elastodynamics. $\rightarrow$ \textcolor{green}{finite-element implementation}
        \end{itemize}
    \end{itemize}
\end{frame}

\begin{frame}{Stability of PML in the literature}
\pause
    \begin{itemize}
        \item \underline{split-field PML:} 
        \begin{itemize}
        \pause
        	\item Stability analysis using Slowness diagrams and wave fronts.
        	\item Definitions of sufficiant and necessary conditions of stability \cite{Becache_elasto}. 
            \item First order finite difference discretization: Prone to instability in anisotropic media.
            \item Second order discretization: stable \cite{Duru2012}

        \end{itemize}
    \item \underline{unsplit-field PML:} 
        \begin{itemize}
        \pause
            \item Stability proved for electromagnetism (Maxwell's equations).
            \pause
            \item In the context of elastic wave propagation problems: no much informations.
        \end{itemize}
    \end{itemize}
\end{frame}

