\section*{Conclusion}
%% Go back to the results

% Description of equations
The development and the analysis of the PML show two of the main aspects of the perfectly matched layers method: its power to attenuate propagating waves and but also its complexity. 
% Numerical results
Indeed we have seen the full development of the equations required to construct the one dimensional PML. Starting from the simple governing equations of a linear elastodynamic problem such as the equation of motion, the constitutive equation and the strain-displacement relationship, we derived a new problem by introducing the complex coordinates and the stretching function $\lambda$. This development allows us to obtain the strong form of the PML. Going from this formulation, we were able to construct the weak form of the PML by introducing virtual displacement in order to implement these equations using the finite elements method. Using a Newmark temporal integration method, we were capable of construct an implicit scheme on the PML for wave propagation and therefore its attenuation. As we have seen the scheme presents a very accurate behavior: it attenuates completely the wave and without any reflection from the truncation interface (using a appropriate set of parameters that we highlighted in the analysis of the effects of the parameters).  \\
% Stability
The analysis of the numerical stability proves that this implicit scheme is also unconditionally stable and shows the critical value of time step $h$ for the explicit scheme used for the propagation of the wave in the physical medium. \\
%% Further work
% Analysis of the numerical scheme 
However certain points remains unclear in this analysis and we need to focus in detail in how to interpret correctly the results of the numerical damping and the relative periodicity error. They will give us more information about the behavior of the scheme and its stability for the PML but also for the physical medium. \\
In future works we will also include a complete analysis of the computational cost of this method: computational resources and time of execution. This will be a prerequisite before implementing the PML into a finite elements code (Akantu). \\
We will also extend this method to the two and three dimensional cases and for anisotropic medium. The same analysis of complexity, stability and computational cost will also be done in order to provide the perfectly matched layers method for the simulation of wave propagation on unbounded domains with the maximum of information and analysis.  

% Physical solution and well posedness

% extend to more concrete case 2D-3D and anisotropic


