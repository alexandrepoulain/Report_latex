\section*{Conclusion}
%% Go back to the results

% Description of equations
The development and the analysis of the PML show two of the main aspects of the perfectly matched layers method: its power to attenuate propagating waves and but also its complexity. 
% Numerical results
Indeed we have seen the full development of the equations required to construct the one dimensional PML. Starting from the simple governing equations of a linear elastodynamic problem such as the equation of motion, the constitutive equation and the strain-displacement relationship, we derived a new problem by introducing the complex coordinates and the stretching function $\lambda$. This development allowed us to obtain the strong form of the PML. Going from this formulation, we were able to construct the weak form of the PML by introducing virtual displacement in order to implement these equations using the finite elements method. Using a Newmark temporal integration method, we were capable to construct two schemes for wave propagation within the PML and therefore ensure its attenuation. A full algorithm was also presented to summarised the main steps of the scheme and the calculation of the different constitutive elements of the PML.
% Stability
After presenting the method that we used to prove the stability of the integration scheme associated with the PML (and therefore the scheme used to simulate the wave propagation within a PML), we have described the different elements that we can extract from the amplification matrix. From this different calculation we also described what can be deduced about the scheme using the example of the standard element. We were able to recorver well known results from the litterature using this stability method. The second step was to apply this method to the PML to see if we can prove the stability. 
The analysis of the implicit scheme associated with the PML has been proved unconditionally stable since the eigenvalues of the amplification matrix remained in the unit circle regardless of the time step considered. The explicit scheme was demonstrated to be conditionally stable and we were able to present the critical value of time step $h$ for which the scheme becomes unstable. Also, an important feature of PML was highlighted with this analysis: the attenuation introduced within the PML permits to postpone the value of the critical time step to larger value. Since the calculation and inversion of the effective stiffness matrix needed in the algorithm is expensive, one can prefer to introduce less time steps within the PML. In fact the analysis of the periodicity error showed that the increase of the time step parameter leads to a larger error but the PML is not designed to model properly and precisely ave propagation. Its usefulness is to attenuate incoming waves. Therefore as much as the wave does not come back in the domain of interest, the accuracy of wave propagation in the PML does not matter.\\
Even if this report was mainly focused on the stability analysis of this particular formulation of the PML, we included a section dedicated to numericak results on two test cases. The first one, the bar test case, is a simplification of the two dimensionnal problem to a one dimensionnal view. The second test case is well known from the geophysist: the Lamb's test. It represents the propagation of seismic wave within a two dimensionnal medium and involves $3$ kinds of waves: P-, S- and Rayleigh waves. The PML has shown the ability to attenuate with efficiency the waves involved in both test cases. Also the reader and the user of the PML scheme implemented within Akantu is strongly encouraged to take profits of the tunable ability of the PML to fit its objectives. Indeed changing the parameters of the PML such as the attenuation coefficients, the time step parameter or the lengths of the elements of the PML can impact the results and a careful choice needs to be made by the user. Of course the same dilemma between performance and accuracy has to be settle. For the PML, the accuracy has to evaluated in term of the reflection of the wave due to the truncation interface.\\     
%% Further work
% Analysis of the numerical scheme 
In future works, a complete analysis of the computational cost of this method needs to be done: computational resources and time of execution. The explicit scheme implemented in Akantu does not support the lumping of the mass matrix reducing drastically the advantage of this latter comparing to the implicit scheme. Introducing the lumping of the mass matrix of the PML will permit the scheme to not calculate the inverse bu only to make a simple matrix-vector calculation. To take profits of the efficiency of this explicit scheme, it will be interesting to implement a coupling method within Akantu. This will permit to choose different time steps and also different integration method for the medium of interest and for the PML. A similar approach has been developped in \cite{Brun2016} using Lagrange multiplier to ensure the continuity of the velocity at the interface. The implementation of such method within Akantu can be a solution to reduce the cost of the computation of the PML by reducing the number of time steps in the PML. This method will be able to take benefit of the ability of the PML to postpone th critical value of time step as we have shown in the stability analysis. \\
Perfectly matched layers have been developped for three dimensionnal problem \cite{Basu2008}. The implementation in Akantu and the stability of the 3D PML can also constitute a following of this work. The field of research concerning the development and analysis of perfrectly matched layer is constantly evolving and its applications can lead to the development of other field of research in geosciences. Since Akantu is open-source library, we strongly encourage the reader to take a look at the code and use the examples concerning the PML to reproduce the results presented in this report. They can also add their own contribution and develop the code or use the already present features in their own projects.       

% Physical solution and well posedness

% extend to more concrete case 2D-3D and anisotropic


